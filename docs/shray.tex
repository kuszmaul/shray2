\documentclass{article}

\usepackage[utf8]{inputenc}
\usepackage{adjustbox}
\usepackage{algorithm}
\usepackage[noend]{algpseudocode}
\usepackage{amsfonts}
\usepackage{amsmath}
\usepackage{amssymb}
\usepackage{amsthm}
\usepackage{booktabs}
\usepackage{caption}
\usepackage{fullpage}
\usepackage{hyperref}
\usepackage{listings}
\usepackage{multicol}
\usepackage{pgfplots}
\usepackage{pgfplotstable}
\usepackage{subcaption}

\DeclareMathOperator{\divi}{div}
\DeclareMathOperator{\md}{~{\mathrm{mod}}~}

\setlength{\parindent}{0pt}
\pgfplotsset{compat=1.17}
    

\begin{document}

\section*{Introduction}

Shray is a distributed shared memory layer tailored for applications operating on arrays.
The API is kept simple so it can be integrated into code-generators. We use the following 
terminology to describe arrays: an $n_1 \times \cdots \times n_d$ array has dimension $d$,
and extent $n_l$ along the $l$th dimension. So 

\[
\begin{pmatrix}
    0 & 1 \\
    2 & 3 \\
    4 & 5 \\
\end{pmatrix}
\]

has dimension two, extent three along the first dimension, and extent two along the second 
dimension. 

\section{Application programmer interface}

\begin{lstlisting}
extern bool ShrayOutput;
\end{lstlisting}

True for exactly one node. Useful to output results only once. 

\begin{lstlisting}
void ShrayInit(int *argc, char ***argv);
\end{lstlisting}

The first statement to be called in your application. The arguments 
\texttt{argc}, \texttt{argv}, are pointers to the commandline arguments passed from 
\texttt{main}. 

\begin{lstlisting}
void *ShrayMalloc(size_t firstDimension, size_t totalSize);
\end{lstlisting}

This function is used to make a distributed allocation. If you would like to allocate an 
array \texttt{TYPE *A} of size $n_1 \times \cdots \times n_d$, then \texttt{firstDimension}
is $n_1$, and \texttt{totalSize} is $n_1 \cdot \cdots \cdot n_d \cdot \texttt{sizeof(TYPE)}$.

\begin{lstlisting}
size_t ShrayStart(size_t firstDimension);
size_t ShrayEnd(size_t firstDimension);
\end{lstlisting}

These two functions are used for ensuring write-consistency to distributed arrays. See 
Section \ref{consistency}.

\begin{lstlisting}
void ShraySync(void *array);
\end{lstlisting}

This function is used to ensure read-consistency to distributed arrays. See Section 
\ref{consistency}.

\begin{lstlisting}
void ShrayFree(void *address);
\end{lstlisting}

Frees memory allocated by \texttt{ShrayMalloc}.

\begin{lstlisting}
void ShrayReport(void);
\end{lstlisting}

Gives information on the number of synchronisations and communicated bytes if Shray is compiled
with macro \texttt{PROFILE} defined.

\begin{lstlisting}
void ShrayFinalize(void);
\end{lstlisting}

The last statement to be called in an application.

\section{Environment variables}

The underlying implementation models a cache. You need to set two environment variables:
\texttt{SHRAY\_CACHESIZE} which is the extra memory each node is allowed to use for 
communication, and \texttt{SHRAY\_CACHELINE} which is the number of 4KB pages in a cacheline. 

\medskip

So for good (decent) performance, a cache-friendly 
algorithm is necessary. That means no strided accesses, and try to tile your loops if possible. 

\section{Consistency model}\label{consistency}

A write to a distributed array $A$ of size $n_1 \times \cdots \times n_d$ at index 
$(i_1, \cdots, i_d)$ is legal, if and only if 
$\texttt{ShrayStart(n1)} \leq i_1 < \texttt{ShrayEnd(n1)}$.

\medskip

A read to a distributed array $A$ is legal, if the read and write are separated by a 
\texttt{ShraySync(A)}. A read is guaranteed to have completed after any \texttt{ShraySync}
(not necessarily on $A$ itself). This can be useful when reusing an allocation. 
Suppose \texttt{f(in, out)} reads from in, and writes to out,
and you do not need \texttt{in} afterwards anymore. It is then safe to reuse \texttt{in} after

\begin{lstlisting}
    f(in, out);
    ShraySync(out);
\end{lstlisting}

\section{Example}

This is what the Hello World of parallel computing, a pointwise matrix addition, looks like.

\begin{lstlisting}
#include "../include/shray.h"

/* Initializes n x m matrix to its row-major index. */
void init(double *matrix, size_t n, size_t m)
{
    for (size_t i = ShrayStart(n); i < ShrayEnd(n); i++) {
        for (size_t j = 0; j < m; j++) {
            matrix[i * m + j] = i * m + j;
        }
    }
}

/* C = A + B */
void matmulAdd(double *A, double *B, double *C, size_t m, size_t n)
{    
    for (size_t i = ShrayStart(n); i < ShrayEnd(n); i++) {
        for (size_t j = 0; j < m; j++) {
            C[i * m + j] = A[i * m + j] + B[i * m + j];
        }
    }
}

int main(int argc, char **argv)
{
    ShrayInit(&argc, &argv, 40960);

    size_t n = 1000;
    size_t m = 2000;

    double *A = ShrayMalloc(n, n * m * sizeof(double));
    double *B = ShrayMalloc(n, n * m * sizeof(double));
    double *C = ShrayMalloc(n, n * m * sizeof(double));

    init(A, n, m);
    init(B, n, m);
    ShraySync(A);
    ShraySync(B);

    matmulAdd(A, B, C, n, m);
    ShraySync(C);

    for (int i = 0; i < 10; i++) {
        printf("C[%d] = %lf. Should be %d\n", i, C[i], 2 * i);
    }

    ShrayFree(A);
    ShrayFree(B);

    ShrayFinalize();
}
\end{lstlisting}

There is no utilisation of \texttt{ShrayOutput} to highlight that every thread reads the
exact same \texttt{C[i]}, not simply the ith element owned locally. 

\section{Warning!}

The implementation catches the SIGSEGV signal, so you cannot use a signal-handler that 
catches this yourself!

\end{document}
