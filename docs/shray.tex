\documentclass{article}

\usepackage[utf8]{inputenc}
\usepackage{adjustbox}
\usepackage{algorithm}
\usepackage[noend]{algpseudocode}
\usepackage{amsfonts}
\usepackage{amsmath}
\usepackage{amssymb}
\usepackage{amsthm}
\usepackage{booktabs}
\usepackage{caption}
\usepackage{fullpage}
\usepackage{hyperref}
\usepackage{listings}
\usepackage{multicol}
\usepackage{pgfplots}
\usepackage{pgfplotstable}
\usepackage{subcaption}

\DeclareMathOperator{\divi}{div}
\DeclareMathOperator{\md}{~{\mathrm{mod}}~}

\setlength{\parindent}{0pt}
\pgfplotsset{compat=1.17}
    

\begin{document}

\section*{Introduction}

This software is based on the Shray distributed shared memory layer from Andrius Šilinskas'
Bachelor's thesis. It is (going to be) simplified and optimised for array programs with 
good locality of reference. It is the aim that this can be used for code generators dealing
with arrays. 

\section{Related work}

\section{Distributed shared memory}

A

\section{Distribution}

\section{Consistency}

There are two forms of ownership: write ownership, and read ownership. If we have $p$ processors
total, and an array $A$ whose first dimension is $n$, processor $s$ may write to 
$[A[s \cdot \lceil \frac{n}{p} \rceil], A[(s + 1) \cdot \lceil \frac{n}{p} \rceil][$. So 
$A$ is distributed block-wise along its first dimension. 

\medskip

% Explain this more clearly with a picture. 

There is a problem with this. Suppose $P(0)$ owns four elements, and pages contain three 
elements. When $P(1)$ reads the fourth element, it fetches the entire page, so also elements
five and six. We also have to make the entire page available for reading. So we must ensure 
that $P(0)$ also has those elements. For this reason the read ownership consists of all pages
containing at least one element owned for writing. After writing, each processor must exchange
this data with its neighbours. 

\end{document}
